
\documentclass[12pt]{article}
\usepackage{amsmath}
\begin{document}

\section*{Tutorial 01 Solutions}

\begin{enumerate}
    \item \textbf{Solution.} Let $x$ be the rate of the ship in still water and $y$ be the rate of the current. 
    Since rate times time is distance, we can write the system of equations:
    \[
    4(x + y) = 60, \quad 5(x - y) = 60.
    \]
    Solve the system:
    \[
    20(x + y) = 300, \quad 20(x - y) = 240.
    \]
    Adding these gives $40x = 540$, so $x = 13.5$. Substituting $x = 13.5$ back, we find $y = 1.5$. 
    The rate of the ship is $x = 13.5$ km/h, and the rate of the current is $y = 1.5$ km/h.

    \item \textbf{Solution.} 
    To match up the sets, we evaluate each set:
    \begin{enumerate}
        \item $\{x^2 : x \in \mathbb{Z}\}$ is the set of integers divided by $2$.
        \item $\{\cos^2(x) + 1 : x \in \mathbb{R}\} = [1, 2]$.
        \item $\mathbb{Z} \cap \{x \in \mathbb{R} : -\frac{1}{2} \leq x \leq \frac{9}{2}\} = \{0, 1, 2, 3, 4\}$.
        \item $\{x - \frac{1}{2} : x \in \mathbb{Z}\} \cup \mathbb{Z}$ is the set of integers and half-integers.
        \item $\{|x| : x \in \mathbb{Z}, -4 \leq x \leq 4\} = \{0, 1, 2, 3, 4\}$.
        \item $\{2(x - 1) : \frac{3}{2} \leq x \leq 2\} = [1, 2]$.
    \end{enumerate}
    Thus, the sets match as follows: 
    (a) $\leftrightarrow$ (d), (b) $\leftrightarrow$ (f), (c) $\leftrightarrow$ (e).

    \item \textbf{Solution.} To prove $A \cap B = A$ when $A \subset B$:
    \begin{enumerate}
        \item By definition of intersection, $A \cap B \subset A$.
        \item Since $A \subset B$, any $x \in A$ is also in $B$, so $A \subset A \cap B$.
    \end{enumerate}
    Combining, $A \cap B = A$.

    \item \textbf{Solution.}
    For each case:
    \begin{enumerate}
        \item $f : \mathbb{N} \to \mathbb{N}$ defines a surjective but not injective function.
        \item $f : \mathbb{Z} \to \mathbb{Z}$ defines a bijective function.
        \item $g : \mathbb{Z} \to \mathbb{Z}$ is neither injective nor surjective.
        \item $g : \mathbb{Z} \to \mathbb{Z}$ is surjective but not injective.
        \item $g : \mathbb{R} \to \mathbb{R}^2$ is injective but not surjective.
        \item $h : \mathbb{Q}^+ \to \mathbb{N}$ does not define a function due to non-uniqueness.
    \end{enumerate}
\end{enumerate}

\end{document}
